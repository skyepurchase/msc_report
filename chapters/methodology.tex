\chapter{Methodology}

\section{Steering Clear Environment}
\label{steering-clear}

The setup of this environment follows \cite{steering-clear}.
The model to steer is a 4-layer multi-layer perceptron (MLP) with residual connections \cite{resnet} across all layers.
After the MLP, a layernorm \cite{layernorm} and single layer classifier is added.
All non-linearity throughout the model is gaussian error linear unit (GeLU).
The hidden layers follow 512-512-256-512 architecture regardless of dataset specifics.

\subsection{Dataset}
\label{steering-clear:dataset}

To control the behaviour of model and the steering approaches a synthetic dataset is used.
Each dataset sample consists of $m$ ``attributes" which can take 8 possible discrete values.
Each discrete value is represented by an ``anchor'' vector $\vmu_i \in \mathbb{R}^8, i \in \{1,8\}$ sampled from a gaussian distribution $\mathcal{N}(\vzero, 1)$.
To simulate real-world conditions gaussian noise is added to the samples from $\mathcal{N}(\vzero, 0.1)$.
This does mean the values are generally highly seperable.

The dataset comprises of $n$ input-output vectors where the input vector is the concatenation of $m$ 8-dimensional vectors.
Thus, an input vector has length $8m$ and the target vector has length $m$.
\citet{steering-clear} carry out a range of experiments for $m \in \{60, 90, 120\}$ but always use 8 values represented by 8 dimensional vectors.
They take a sample of $2,000,000$ i.i.d samples but due to memory constraints only $500,000$ are used in this project.
No test set is used in either however an 80:20 split for training and validation set is used for identifying the best performing model.

\subsection{Pre-training}

The MLP model is trained on the $500,000$ training samples for 50 epochs using Adam \cite{adam} with a learning rate of $0.001$.
As per \citet{steering-clear} a cross entropy loss is used to train the model.
The model that achieves the best validation loss is saved and used for the steering task.

Regardless of exact epochs, learning rate or optimiser the best performing model should achieve close to $100\%$.
Models used for the presented results achieved $\sim 99\%$.

\subsection{Steering Task}

The task is to successfully steer a model to always predict a specific value for a specific attribute.
For example, the goal would be steer attribute $3$ towards value $\vmu_1$.
\citet{steering-clear} carry out three experiments to steer one, two or three attributes simultaneously.
Instead, this reproduction will focus on steering only one attribute at a time.

As the attribute anchors are generated randomly there is no dataset bias towards any particular value.
For this reason all attributes are steered towards value $\vmu_1$.

In addition to the dataset generate, $4096$ are generated as a training set for the steering approaches and a further $1000$ are generated as a test set.
This is repeated 20 times to get an average metric across steering approaches.

\section{Prompt Pairs Environment}
\label{prompt-pairs}

\subsection{dataset}

\subsection{Steering Task}
