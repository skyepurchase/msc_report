\chapter{Introduction}
\label{ch:introduction}

\emph{In this chapter an overview of the project and this report is detailed.}
\emph{This includes what the project involves as well as why this is a worthwhile project.}
\emph{The motivation for the project will be briefly discussed along with the benefits of the analysis carried out.}

\emph{Related work to this project, including projects which precede this one and projects which cover separate but related issues, is presented.}
\emph{The differences between this project and the related work is explained.}

\emph{Finally the contributions which this project make to the field are detailed.}
\emph{This contributions are justified during the results chapter \Sref{ch:results} and reiterated in the conclusion chapter \Sref{ch:conclusion}.}

\section{Motivation}

In recent years the use of machine learning (ML) models has rapidly increased across multiple sectors of life.
In particular the rise in deep learning models following the release of Alexnet \citep{alexnet} has introduced new problems with the use of ML.
This has lead to rapid development in field resulting  model capabilities surpassing human capabilities in an increasing range of activities \citep{dynabench, gpt-5, grok-4, hle}.
With the introduction of large language model (LLM) chatbots starting with ChatGPT \citep{chatgpt}, ML models are increasingly becoming part of everyday life.

There are already cases of serious harm to users \citep{c.ai, psychosis} and the potential for humanity scale risk posed by increasingly capable systems has been posed \citep{survellience, deepfakes, disempowerment}.
There are many approaches to try and solve these problems ranging from law and policy to model training techniques and dataset design.

The field of preventing risks posed by ML is known as AI (artificial intelligence) safety.
\textbf{A primary goal of AI safety is to alter models to be more ``helpful'', ``honest'', and ``harmless'' without effecting their performance}.
A promising avenue of AI safety, especially for LLMs, is \emph{steering adaptors} a subset of representation engineering \citep{steering-taxonomy}.
The basic concept behind steering adaptors is to \emph{manipulate the internal representation space} of models.
By assuming certain \emph{directions within representation space represent understandable concepts} it is possible to perturb a models vector representation towards a target concept.
This idea has already been demonstrated by \citet{word2vec} in word embedding models.

There is large amounts of anecdotal evidence of steering adaptors successfully changing model behaviour with increasing studies looking into the theory \citep{steering-clear, steering-theory} and effectiveness \citep{steerability, steering-taxonomy} of steering adaptors on modern models.
There, however, continue to be a lack of large scale studies comparing a range of adaptors in realistic settings with varying input size and hyperparameter choice.

This project and report aim to remedy this and provide more quantitative and qualitative analysis of common steering techniques on LLMs.
The project will hopefully provide insight into the effectiveness of these techniques as well as the suitability of a selection of suggested metrics.

\subsection{Benefits of Steering Vectors}

Unlike other techniques to change a models behaviour such as reinforcement learning \citep{rl, rlhf} or finetuning \citep{lora} steering adaptors require far less compute and data as they do not require changing the many parameters of the chosen model.
Once implemented they are efficient to run during inference and have been shown to have low impact on model capabilities \citep{steering-wo-ss, sea}.
Importantly, these techniques are still compatible with other alignment techniques working to provide additional adjustments.

By nature they inherently provide some interpretability by demonstrating how certain model representations effect the output.
This, in turn, allows for more precise control over model behaviour than other techniques with potential direct feedback between the user and the model behaviour.
However, these techniques frequently require full access to the model's parameters and architecture.

\section{Research Questions}
\label{sec:questions}

This project aims to provide a detailed, quantitative comparison of steering adaptors in realistic settings.
This work follows the work on analysing the generalisability of steering vectors in \citet{steerability} and the survey of current representation engineering techniques from \citet{steering-taxonomy}.
The approach used in this project extends the analysis of steering example size on adaptor performance in the toy setting proposed by \citet{steering-clear} to the natural language setting using LLMs.
Furthermore, a comparison of the new affine concept editing \citep[ACE]{ace} adaptor is made against existing adaptors.
The three research questions therefore focus on reproducing prior results and providing new results and analysis.

\begin{enumerate}[nolistsep]
    \item Are the findings of \citet{steering-clear} reproducible and sound? Where does the adaptor proposed by \citet{ace} fit into their analysis?
    \item How do the number of examples effect the performance of steering adaptors? Does this match the results in the toy setting of \citet{steering-clear}? What metrics provide meaningful insight into steering performance?
    \item How can the success of steering adaptors be quantified in the natural language setting?
\end{enumerate}

Overall this projects aims to provide more insight into when steering methods work focusing on the number of steering examples.
Additional discussion on the shortcomings of the steering approaches are provided throughout as well as suggestions for real world use.

\section{Compute Environment}

The project was written and programmed on my personal laptop.
This is a Dell Inspiron 15'' laptop running Arch Linux and i3 window manager.
The laptop has 16GB of memory with an 8 core Intel 13$^{th}$ generation i7 processor.

Preliminary tests of the experiments run in this project were done on this laptop.
However, the official results were run on \href{https://www.runpod.io/}{RunPod}, which provides a suite of possible GPU environments.
For all the results presented in \Sref{ch:results} the RTX 2000 Ada environment was used.
This environment includes a single RTX 2000 Ada GPU with 16GB of VRAM, 31GB of RAM, and 6 virtual CPUs.
The cost to run this environment was \$0.24 per hour.
The funding for the compute environment was provided by my supervisor through the \href{https://www.matsprogram.org/}{ML Alignment and Theory Scholarship} which he is part of.

Occasional use of the free \href{https://colab.research.google.com/}{Google Colab} environment was used to verify the experiments ran at scale.
No results presented in this report were generated from these runs.

\section{Generative AI Disclosure}

This project aims to steer generative AI models (genAI) such as GPT 2 \citep{gpt-2} and thus genAI was used in generating the raw results presented in \Sref{ch:results}.

However, outside of the direct prompting of agents to analyse the steering adaptors presented in \Sref{ch:background}, no generative AI was used throughout the project or report.
There are two exceptions:
\begin{itemize}[nolistsep]
    \item Generating a dataset of prompt-completion templates detailed in \Aref{app:prompt-pairs}.
    \item Generating a random point cloud in Tikz for \Fref{fig:caa}.
\end{itemize}

Spellcheck was performed by Neovim and its built in spellchecker with no use of AI tools such as Grammarly.
All citations were found and generated through Google Scholar.
Assistance for Tikz diagrams was found through \href{https://www.tikz.dev}{https://www.tikz.dev}.

However, with the proliferation of AI in search engines it is not possible to state that genAI did not influence any of the \LaTeX~or code snippets.
I endeavoured to not use genAI as much as possible throughout the last 3 months.

\section{Related Work}

\smalltitle{\citet{steering-clear}}
aim to analyse steering in a toy environment where they are able to control the representation density within the model.
They compare a range of steering techniques \citep{caa, reft, mimic} against each other in a controlled setting to focus on the effect of steering example size.
Inspired by the low-rank representation finetuning \citep[LoReFT]{reft} (A steering adaptor which steers representations in a projected, low-rank space) they introduce their own technique, low-rank representation steering (LoReST), and demonstrate competitive performance to the other techniques.

To verify this work, this thesis reproduces the same toy environment described in \Sref{sec:steering-clear} and validates the conclusions drawn.
Additionally, this reproduction introduces the affine concept editing adaptor proposed by \citet{ace} to compare against the adaptors used in \citet{steering-clear}.
This thesis then expands the analysis to LLMs and demonstrates similar adaptor performance in relation to the number of training examples suggesting the same preference for low-rank methods in the large data regime and affine methods in the small data regime.
Furthermore, this thesis shows the consistency of LoReST in both data regimes.

\smalltitle{\citet{steerability}}
analyse the generalisation of steering vectors across a range of steering datasets.
They analyse the variability of success and introduce the notion of ``steerability''.
Using this notion they demonstrate that many techniques fail to generalise on certain datasets both in and out of distribution.

The analysis is limited to only contrastive activation addition \citep{caa} which \citet{steering-clear} show is not necessarily the ideal candidate.
Building on their work this project aims to analyse a larger range of techniques sampled from \citet{steering-clear}.
Furthermore, the properties of training datasets is analysed in more depth to determine which properties cause steering techniques to fail.

In this thesis, rather than use model written evaluations \citep{mwe} a new set of steering datasets is generated with more fine grain control.
The construction of these datasets is described in \Sref{sec:prompt-pairs}.

\smalltitle{\citet{steering-taxonomy}}
present a full taxonomy of current steering vector approaches (more generally \emph{representation engineering}).
The paper covers a range of topics within representation engineering that have been carried out by the community.
These focus on the types of adaptors used, the prompting framework, linear vs. non-linear adaptors, the concepts that are steered, etc.

This thesis continues these comparisons by analysing the effect of the dataset and number of steering examples used in the large language model setting.
It is impractical to expand the experiments to all the approaches described in \citet{steering-taxonomy} due to time constraints so only those discussed in \citet{steering-clear} are analysed.


\smalltitle{Sparse autoencoders as steering vectors}
There are multiple papers that utilise sparse autoencoders (SAEs) as steering vectors \citep{sae-improved, sae-steering, icl-sae}.
SAEs work by transforming the dense model representations into a high-dimensional, sparse vector space.
This allows individual dimensions to represent unique, human-understandable concepts effectively ``decoding'' the model representation.
These approaches utilise this fact, that SAEs decode high level concepts from the models intermediate representation to steer the model towards or away from said concepts.
This process works by ``reversing'' the SAE procedure and converting concepts into potential representation perturbations.

This thesis, in contrast, uses the SAE features to evaluate models on free-text responses rather than utilising SAEs to steer the model.
The SAE features provide a metric to evaluate how well the models internal representation has been effectively steered.
A detailed description of SAEs is provided in \Sref{sec:sae} and their use in this project is described in \Sref{sec:prompt-pairs}.

\section{Contributions}
\label{sec:contributions}

In answering the questions posed in \Sref{sec:questions} this project provides the following contributions
\begin{itemize}[nolistsep]
    \item Verification of the results presented by \citet{steering-clear} in their toy experiment.
    \item Comparison of contrastive activation addition \citep[CAA]{caa}, affine concept editing \citep[ACE]{ace}, LoReFT and LoReST on large language models.
    \item In depth analysis on the effect of the number of steering examples on the adaptor performance.
    \item A set of promising LLM steering metrics along with qualitative analysis of the steered model output.
\end{itemize}

All code for the project is available at \href{https://github.com/skyepurchase/msc_thesis}{this code repository} and all the \LaTeX~including diagrams is available at \href{https://github.com/skyepurchase/msc_report}{this, separate, code repository}.
