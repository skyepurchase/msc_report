\chapter{Introduction}

\section{Related Work}

\smalltitle{\citet{steering-clear}}
aim to analyse steering in a toy environment where they are able to control the representation density within the model.
They compare a range of steering techniques \cite{caa, reft, mimic} against each other in a controlled setting to evaluate the benefits and drawbacks of each approach.
Inspired by LoReFT \cite{reft} they introduce their own technique LoReST and demonstrate competitive performance to the other techniques.

This project reproduces a sample of plots from Figure 1 using the same toy setup described in \Sref{steering-clear}.
In addition to the techniques used in the original paper the reproduction also analyses the behaviour of \cite{ace}.

This project aims to expand the analysis carried out by \citet{steering-clear} to reproduce the same effects in large language model systems.
Additionally, the relationship between the negative and positive training examples is analysed to gain a better insight as to when steering approaches fail.

\smalltitle{\citet{steerability}}
aim to analyse the generalisation of steering vectors across a range of steering datasets.
They analyse the variability of success and introduce the notion of steerability.
Using this notion they demonstrate that many techniques fail to generalise on certain datasets both in and out of distribution.

The analysis is limited to only contrastive activation addition \cite{caa} which \citet{steering-clear} show is not necessarily the ideal candidate.
Building on their work this project aims to analyse a larger range of techniques sampled from \citet{steering-clear}.
Furthermore, the properties of training datasets is analysed in more depth to determine which properties cause steering techniques to fail.

Rather than use model written evaluations \cite{mwe} a new set of steering datasets is generated with more fine grain control.
The construction of these datasets is described in \Sref{prompt-pairs}.

\smalltitle{\citet{steering-taxonomy}}

\section{Contributions}
